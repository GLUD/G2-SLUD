%********************************************************
% To change the recipient, look at the bottom.
%
% To generate pdf, use "pdflatex aletter".
%
% This template does not work with the "latex" command,
% use "pdflatex". (Reason: the file type of the logo.)
%
% Hint: When you see "??" instead of the total number  
% of pages, try to compile second time using pdflatex.
%
% @author: Michal Serkieza 2010
%
% [This version 1.1 prepared by EL 2010-11-02]
%********************************************************

% Use document class "aletter.cls" (for letters):
\documentclass[11pt, twoside]{aletter}
% Concerning the parameters:
% - use font size other than "11pt" only at your own risk
% - "twoside" differs from "oneside" only as follows: if 
%   multiple copies are generated (with multiple 
%   instances of "\makecopy"), then "twoside" makes each 
%   copy occupy an even number of pages    

% ISO 8859-1 character encoding is assumed
\usepackage[utf8]{inputenc}
\usepackage[T1]{fontenc}

% Choose one of the supported texts for the Aalto logo:
\logoseries{logog2} % "Aalto University" [the default] 
%\logoseries{Aalto_SST} 

% Choose one of the Aalto logo variants: 
%\fixlogo{1} % blue ! [the default]
%\fixlogo{2} % blue "
%\fixlogo{3} % blue ?
%\fixlogo{4} % red !
%\fixlogo{5} % red "
%\fixlogo{6} % red ?
%\fixlogo{7} % yellow !
%\fixlogo{8} % yellow "
%\fixlogo{9} % yellow ?
%\varylogo   % pseudorandom, varies page-by-page

\headtitle{Grupo GLUD}
\headsubtitle{Universidad Distrital Francisco Jos\'e de Caldas}
\date{\today}
%\hidepagenumbers

% On the footer text definitions:
% - some Aalto-wide definitions are in "aletter.cls"
%   and thus not visible here 
\enSchool{Universidad Distrital}
\enDepartment{Facultad de Ingenier\'ia}
\enPerson{Ph.D. student Artturi Björk}
\fiSchool{Kauppakorkeakoulu}
\fiDepartment{Taloustieteen laitos}
\fiPerson{Tohtorikoulutettava Artturi Björk}
% - the commands below have the following implicit
%   'chained side effects':
%   \<xy>Univ<Abc> calls even \<xy>Schl<Abc>,
%   \<xy>Schl<Abc> calls even \<xy>Dept<Abc>, and
%   \<xy>Dept<Abc> calls even \<xy>Pers<Abc>
% - therefore, the definitions should occur in the 
%   following order: Univ -> Schl -> Dept -> Pers
\enDeptPostA{P.O.~Box 21240}
\enDeptVisitA{Arkadiankatu 7}
\enDeptVisitB{Helsinki, Finland}
\enPersTel{Tel +358 50 3707121}
\enDeptFax{}
\enPersFax{}
\enSchlWeb{biz.aalto.fi/en}
\enDeptWeb{economics.aalto.fi/en}
\enPersWeb{https://sites.google.com/site/artturibjork}
\enPersEmail{artturi.bjork@gmail.com}
%
\fiDeptPostA{PL 21240}
\fiDeptVisitA{Arkadiankatu 7}
\fiDeptVisitB{Helsinki}
\fiPersTel{Puhelin 050 3707121}
\fiDeptFax{}
\fiPersFax{}
\fiSchlWeb{biz.aalto.fi}
\fiDeptWeb{economics.aalto.fi}
\fiPersWeb{https://sites.google.com/site/artturibjork}
\fiPersEmail{artturi.bjork@gmail.com}

% Choose one (of the footers defined in "aletter.cls"):
%\deptfooter     % departmental [the default] 
%\persdeptfooter % personal (naming the department too) 
%\persschlfooter % personal (naming the school too)

% A-EditFoot: To further experiment with customizing the 
% footer texts, select Finnish (see below), and activate 
% the following line and view the file it refers to: 
%\input{a-editfoot}

\title{HOLAAAAAAAAAAe}
%\title{} % An alternative to the previous line

% In the case of the signature, some alternatives are:
% (A) A simple signature:
\signature{Laydi Viviana Bautista\\Directora Grupo GNU Linux Universidad Distrital FJC}
% (B) Even extensive sender information can be included 
% here (especially if it is not included in the footer): 
%\signature{Dr.~Alice Allen\\
%Coordinator of XYZ Programme\\
%\\
%Tel +358 9 470 99999\\
%Fax +358 9 470 88888\\
%\url{firstname.lastname@tkk.fi}\\
%\url{http://users.ics.tkk.fi/name/} % No newline any more
%}
% (C) When the letter is signed by two people:
%\signature{\begin{tabular}{@{}p{62mm}@{}l@{}}
%Mr.~Bob Benson & Ms.~Carol Clark\\
%President      & Director
%\end{tabular}
%}

\begin{document}
%\selectlanguage{finnish} % Suppress this to select English

\renewcommand*{\makecopy}[2]{
\begin{letter}{#2}
\opening{#1}
\textbf{REF: Solicitud patrocinio para la Semana Linux XVI Universidad Distrital FJC.}
% -- Espacio para contenido

\\

Como comunidad, nos contactamos con ustedes con el objetivo de exponer la solicitud de patrocinio a su compañía, el próximo % Completar fecha
daremos inicio a la \textit{XVI Semana Linux} de la Universidad Distrital Francisco Jos\'e de Caldas.

\\
La Semana Linux es un evento anual de visibilidad nacional, organizado por el \textit{Grupo GNU Linux de la Universidad Distrital} (GLUD), en el cual se da a conocer el Software Libre y sus ventajas para el progreso tecnológico de la sociedad, allí convergen estudiantes, profesores, expertos y usuarios finales interesados en familiarizarse con el uso del Software Libre y su filosofía.

\\
La temática de la Semana Linux para este año es \textit{Inteligencia Artificial}, durante toda la semana se darán conferencias, talleres y actividades acerca de diferentes temas enfocados a promover la tecnología, software, cultura y ciencia libre.

\\
Contaremos con conferencistas del sector privado y público, además de diferentes Universidades de la ciudad. Así, nos complace invitar a \textit{Nombre Empresa} como patrocinador del evento haciéndolo participe en las actividades de nuestro evento. Con su patrocinio ofrecemos:

\begin{itemize}
    \item Espacio de dos horas para conferencias que deseen presentar como compañía, todo enfocado en uso de herramientas y Software Libre.
    \item Exposición de su marca, logo y publicidad en el sitio del evento.
    \item Publicación de su logo representativo en los banners y posters oficiales del evento.
    \item Registro de asistentes al evento para b\'usqueda de talentos que puedan servir en las b\'usquedas de staff de su compañía.
\end{itemize}

Solicitamos el apoyo económico de \textbf{500 USD} los cuales se utilizarán de la siguiente manera:

\begin{table}[htbp]
\begin{center}
\begin{tabular}{|l|l|}
\hline
Descripción & Costo \\
\hline \hline
Impresión de productos litográficos como escarapelas, \\folletos, libretas, cuadernos, volantes y adhesivos\\ estampados con el logo de los patrocinadores, de la \\Universidad y de la Universidad & 250 USD \\ \hline
Refrigerios para los conferencistas durante toda la semana & 100 USD \\ \hline
Libros para brindar en las conferencias & 150 USD \\ \hline
\end{tabular}
%\caption{}%Tabla muy sencilla.}
%\label{tabla:sencilla}
\end{center}
\end{table}

Quisi\'eramos poder contar con su apoyo y asistencia al evento para tener la oportunidad de compartir con ustedes conferencias, conocimiento y oportunidades que converger\'an en el mismo.
%\input{contenido2.tex}

\vspace*{0mm} % Extra vspace between text and greeting 
%\vspace{-8mm}\closing{} % An alternative closing
\closing{Gracias,
\vspace{0mm} % Extra vspace between greeting and signature
}
\vspace*{0mm} % Extra vspace after signature

%\extra{CC}{Dr.~Dave Davies} % You may want to suppress this
%\extra{ENCL}{Statistics\\Diagrams} % You may want to suppress this

\end{letter}
} % End of the (re-)definition of "\makecopy" 

% Having multiple "\makecopy" instances produces multiple letters:
\makecopy{Ingeniero,}{Nombre\\Direcci\'on\\Dato Adicional}
%\makecopy{Dear Tiina,}{Dr. Tiina Teekkari\\Jämeräntaival 99 A 1\\02150 Espoo}

% A-AddrBook: To experiment with some higher-level commands
% built on top of "\makecopy" and offering some rudimentary
% support for creating an address book, activate the following 
% line and view the file it refers to: 
%\input{a-addrbook}

\end{document}
